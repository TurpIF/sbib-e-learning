\documentclass[conference]{./sty/IEEEtran}
% \documentclass[a4]{article}

% pour les accents utilisés en français
\usepackage[utf8]{inputenc}
\usepackage[T1]{fontenc}

% pour inclure joliment un algorithme
\usepackage{url}
\usepackage[a4paper,pagebackref,bookmarksnumbered]{hyperref} % ps2pdf
% pour inclure joliment un algorithme
\usepackage{algorithmic}

\usepackage{array}
\usepackage{booktabs}

% Si Latex ne fait pas correctement la césure de certains mots,
% indiquez les emplacements corrects pour les tirets dans ces mots
\hyphenation{op-tical net-works semi-conduc-tor}


\begin{document}

\title{Synthèse bibliographique}


\author{
\IEEEauthorblockN{Jérémy Edert, Mathis Paul, Pierre Turpin}
\IEEEauthorblockA{INSA de Lyon}
}

\maketitle

\tableofcontents

\section{Introduction}

Le e-learning se développe de façon importante dans un grand nombre de
domaines, notamment en entreprise pour la formation continue des personnels.
Dans le cadre du projet de synthèse bibliographique, nous cherchons à étudier
et comparer les différentes solutions existantes dans le domaine des systèmes
de recommendation appliqués au e-learning. \\

L'ensemble des fiches de lecture ci-jointent en annexe nous a permis d'établir
cette synthèse et résument les papiers de recherche lues. Ces articles ont été
repris de différentes conférence et journaux et se concentre sur les trois
thèmes principaux que nous aborderont : la recherche de contenu, l'utilisation
du contexte et des informations de l'utilisateur lors du ciblage, puis
l'application dans le domaine du e-learning. \\

La plupart des articles lus et résumés sont tirés des références de
\cite{DBLP:journals/tlt/VerbertMOWDBD12}. En effet ce dernier regroupe déjà
beaucoup de résultats et de méthodes de recherche contextuelle de documents
pédagogique pour du e-learning. \\

% TODO bullshiter un peu plus

\section{Recherche de contenu}

Le problème de l'application souhaité repose sur la recherche de données et la
proposition de solutions à l'utilisateur. \\

\section{Utilisation du contexte et des informations de l'utilisateur lors du ciblage}

Pour personnaliser et fiabiliser les résultats de recherche pour un
utilisateur, le contexte et les données utilisateurs sont acquis et modifie le
rang des éléments. \\

Le besoin de modéliser proprement et strictement ce genre d'information est
donc devenu nécessaire. Beaucoup de chercheurs étudient et publient des papiers
sur la notion de contexte et sur sa définition la plus exacte possible
\cite{DBLP:journals/tlt/VerbertMOWDBD12},
\cite{DBLP:reference/rsh/AdomaviciusT11},
\cite{DBLP:journals/aim/AdomaviciusMRT11}. \\

\subsection{Connaissance et acquisition du contexte}

Chaque attributs formant le contexte doit avoir sa méthode d'acquisition, de
mise à jour (s'il y en a une) et sa "valeur par défaut" lors de données
manquantes. En effet, toutes ces caractéristiques peuvent être statique ou
dynamique et observable ou peu voir non observable
\cite{DBLP:journals/aim/AdomaviciusMRT11}. Le tableau
\ref{tab:update_observability_context} reprend les caractéristique possible
pour un attribut de contexte. \\

\begin{table}
  \caption{\label{tab:update_observability_context} Information sur le contexte \cite{DBLP:journals/aim/AdomaviciusMRT11}}
  \begin{tabular}{|p{0.20\linewidth}|p{0.20\linewidth}|p{0.20\linewidth}|p{0.20\linewidth}|}
    \hline
    ~ & Totalement observable & Partiellement observable & Non observable \\ \hline
    Statique & Connaissance totale & Connaissance partielle et statique & Connaissance latente \\ \hline
    Dynamique & Connaissance totale dynamique & Connaissance partielle et dynamique & Aucune connaissance \\ \hline
  \end{tabular}
\end{table}

Les attributs statiques sont des simplifications que le système peut se
permettre car la donnée est considérée comme figé dans le temps le long de
l'utilisation de l'application par un utilisateur. Par exemple une date de
naissance, une identité sont des données fixe pour un utilisateur. Lorsque ces
attributs sont pleinement observables, le système peut considérer les connaître
entièrement. C'est les meilleurs cas possibles. S'ils sont partiellement
observable ou peu ou non observable, alors cela implique un retard dans
l'obtention des données plus ou moins grand en fonction de l'observabilité des
attributs. Cependant une fois totalement observés, le système connaît ces
attributs. \\

Les attributs dynamiques sont des données mesurées ayant une certaine durée de
vie. Cette durée dépend de la nature de l'attribut. La localisation
géographique de l'utilisateur n'a pas besoin d'être précise au mètre près. Ce
type d'attribut n'est donc pas à mesurer très frequement et comporte une durée
de vie plutôt importante. Mais l'horloge tout simplement est une donnée à
mesurer chaque seconde par définition. Ces différentes mesures deviennent
possible et une large gamme de type d'attribut peut être obtenu par les
nouvelles technologies inclusent dans les téléphone et appareils portables
\cite{DBLP:conf/wstst/Kurti08}. \\

Lorsque les attributs ont des valeurs dynamiques, l'observabilité totale permet
d'avoir une connaissance de toute la dynamique de la dimension et donc
d'utiliser avec beaucoup plus de précision cette information. On peut dire que
l'observabilité est totale si le temps d'acquisition de la mesure est plus
petit que sa durée de vie. Cette granularité est à définir par le designer de
l'application en fonction de l'usage fait. Un temps d'acquisition plus lent
entraînera alors une observabilité que partielle. La dynamique de l'information
ne pourra être totalement connue et le système devra s'adapter avec des données
manquantes ou anciennes. Pour certains types, les valeurs manquantes peuvent
être inférées et connu \cite{DBLP:journals/prl/TruccoFR99}. Il est à noter
qu'il y a différents niveau d'observabilité partielle. Ces différences ne sont
pas examiner dans cette synthèse. Lorsque la mesure est non observable, aucune
connaissance ne peut être obtenu et le système doit pouvoir fonctionner avec ce
manque. \\

Il existe deux types de dynamique pour les attributs : dynamique passive et
explicite. \\
Un attribut passivement dynamique peut changer sans agissement
directe de l'utilisateur. En effet, dans le cas d'une géolocalisation,
l'utilisateur n'a pas besoin d'agir directement sur les appareils mobiles pour
changer sa position. Cette valeur est mesuré automatiquement par des capteurs
de positionnement, comme le Global Positionning System (GPS) ou le Wi-Fi
\cite{DBLP:journals/tlt/VerbertMOWDBD12}. \\
Les attributs dont la valeur change via l'action directe de l'utilisateur sont
considérés comme dynamique explicite. C'est le cas en général pour toutes les
informations utilisateurs. Cela peut également correspondre à d'autres types de
données comme des filtres et des préférences que l'utilisateur décrit sur
l'application. Ces données sont, de plus, totalement observables et permettent,
dans le cas des filtres et préférences, d'affiner encore plus les résultats de
recherche proposés. Cela accroit donc l'efficacité de l'application quant au
ciblage. \\

\subsection{Les différents types de contexte}

La notion de contexte a été étudier dans différentes sciences (psychologie,
intelligence artificielle, modélisation cognitive, ...) et chacune de ces
sciences expriment un point de vue différents
\cite{DBLP:reference/rsh/AdomaviciusT11}. C'est pourquoi nous ne pouvons pas
donner une liste exhaustive des éléments pouvant définir cette notion. Nous
nous baserons principalement sur un nombre fini d'aspects et d'attributs les
plus utilisés dans le cadre des applications de recommendation avec contexte
pour le e-learning. \\

\subsubsection{Ordinateur et appareils mobiles}
~\\
Connaître les outils qu'à l'utilisateur et utile pour le système de
recommandation. En effet, celui-ci peut ainsi savoir ce qu'il peut recommander.
Cela concerne les capacités réseaux, matérielles et logicielles. \\

Au niveau réseau, l'application peut proposer du contenu externe comme des
documents, des vidéos, des supports de présentation ou tout simplement des
pages web. \\

La capacité matérielle et surtout la différence entre ordinateur et appareils
mobiles est une connaissance importante. Un ordinateur est un outils plus
puissant avec plus de mémoire. L'application peut proposer à l'utilisateur
d'utiliser des logiciels tournant sur ordinateur, ou peut présenter des pages
web lourdes (visualisation 3D par exemple) et peut aussi sugérer un
téléchargement de contenu. \\
Un appareil mobile, même si elle est moins puissante et a moins d'espace de
stockage, possède en revanche une quantité croissante de capteurs utilisable
pour compléter les résultats de recommendation \cite{DBLP:conf/wstst/Kurti08}.
Egalement la présentation de l'application ne sera pas fait de la même manière
sur un ordinateur et sur un appareil mobile à cause de la taille de l'écran. \\

Enfin la connaissance des capacités logicielles permet de rafiner les résultats
en soumettant uniquement des recommendations utilisables pour l'utilisateur.
Par exemple des fichiers \emph{Microsoft Office} ne peuvent pas être ouvert sur
toutes les plateformes. \\

\subsubsection{Géolocalisation}
~\\

\subsubsection{Temps}
~\\

\subsubsection{Activité}
~\\

\subsubsection{Ressources}
~\\

\subsubsection{Utilisateur}
~\\
Pour fiabiliser les résultats de recherche pour un utilisateur, le système doit
prendre en compte les informations de l'utilisateur. \\

\subsubsection{Relations sociales}
~\\

\subsection{Représentation et modélisation du contexte}

\subsection{Algorithmes de recommandations contextuelles}
~\\D'un point de vue algorithmique, les méthodes utilisées pour personnaliser l'expérience utilisateur dans le cadre du e-learning sont très. Elles font intervenir des paramètres de contexte différents, et se répartissent globalement entre celles-faisant appel à des recherches avec filtrage contextuel, à du filtrage collaboratif où reposant uniquement sur des éléments issus directement de la modélisation contextuelle.
\subsubsection{Pré-filtre contextuel}
~\\

\subsubsection{Post-filtre contextuel}
~\\

\subsubsection{Filtrage collaboratif}
~\\Le filtrage collaboratif désigne de façon générale les méthodes recommandant du contenu à un utilisateur à partir d'informations issues du contexte du comportement d'autres utilisateurs. Il s'agit de rechercher des clusters d'utilisateurs aux comportements similaires susceptibles d'apprécier les mêmes contenus, afin de pouvoir recommander à un membre du groupe ce que les autres aiment déjà. \cite{Liou:2014:CPL:2617848.2617854} utilise par exemple les données issues du système de  e-learning des internes d'un hôpital pour répartir les utilisateurs en groupes afin de recommander de manière efficace des actions à effectuer sur la plate-forme. 

\subsubsection{Utilisation directe du contexte}
~\\Certaines solutions se reposent directement sur des événements contextuels  pour proposer du contenu à celui-ci. Les types d'algorithmes utilisés varient alors considérablement d'une solution à l'autre. La solution de formation en entreprise APOSDLE utilise par exemple une représentation à base d'ontologie pour représenter les taches effectuées par un utilisateur lors de son travail habituel ainsi que les compétences nécessaires pour les mener à bien\cite{DBLP:journals/procedia/BehamKLL10}. Cette démarche consiste donc à modéliser précisément les taches qu'ont à effectuer les différents collaborateurs d'une entreprise utilisateurs de la solution ainsi que les compétences qu'elles mettent en jeu. Les relations entre taches et compétences sont également modélisées. L'application est capable d'évaluer le niveau de compétence de chaque utilisateur par rapport à chacune de ces compétences à partir des interactions de celui-ci avec les applications nécessaires à son travail. L'outil peut alors recommander aux employés des collaborateurs experts dans ces différentes compétences, à même de les aider dans leurs missions. Une approche similaire est mise en place dans la solution d'apprentissage de langues étrangères mobile PALLAS \cite{DBLP:conf/wmte/PetersenM06}. La position de l'utilisateur est utilisée pour afficher des messages personnalisés en fonction notamment de la langue étudiée et du niveau de l'utilisateur pour inciter celui-ci à se rendre sur des points d'intérêt. \cite{Kurti:2008:CMS:1456223.1456331} propose une approche similaire en ce que le contexte utilisateur, notamment sa position, sert d'élément déclencheur à la proposition d'activités d'apprentissage sur une solution mobile. Des méthodes à base d'algorithmes génétiques sont également envisageables. En modélisant les cours en ligne d'une université en tant que suite de sections composées d'objets d'apprentissage de nature différentes, \cite{smartECourseRecommander} recommande par exemple aux auteurs de ces cours des nombres idéaux d'éléments comme des paragraphes, graphiques ou exercices à y ajouter. En constatant l'importance des différents modes d'apprentissage dans la capacité des étudiants à assimiler des connaissances, il y est assumé qu'une variété d'items permette d'adapter les cours à ces différents types d'apprentissage . Cette adéquation aux modes d'apprentissage est mesurée pour chaque section de cour par des indicateurs basés sur la fréquence relative des différents types d'objets les uns par rapport aux autres. Un algorithme génétique est ensuite utilisé pour tester successivement différentes compositions de cours afin d'optimiser leur faculté à être efficaces pour un maximum de méthodes d'apprentissage. 

\section{Application spécifique dans le domaine du e-learning}
~\\
\subsection{Les différents types de recommandation}
	~\\L'intérêt de la personnalisation des plate-formes de e-learning en fonction du contexte utilisateur est de produire de façon générale des solutions de meilleure qualité en rendant l'apprentissage plus efficace. Cette personnalisation s'effectue notamment via des méthodes de recommandation qui se différencient avant tout par la nature des objets recommandés. Des approches très différentes ont en effet été imaginées, différents types de recommandations pouvant être utilisés pour une mêm solution. \\
	~\\L'approche la plus courante retenue dans la personnalisation de systèmes d'e-learning est la recommandation de ressources d'apprentissage \cite{DBLP:journals/tlt/VerbertMOWDBD12}. Il peut s'agir d'items complémentaires à la formation comme des exemples ou des éléments exercices ayant pour but de faciliter l'apprentissage aux élèves, comme de cours complets susceptibles de les intéresser. \\
~\\L'on peut également envisager de recommander du contenu non pas à l'étudiant, mais à l'auteur du cours \cite{smartECourseRecommander} dans l'optique que celui-ci soit suffisamment pourvu en ressources variées lui permettant de s'adapter le mieux possible aux différents modes d'apprentissages des lecteurs. \\
~\\Le postulat que l'aide apportée par des collègues est la meilleure stratégie d'apprentissage sur le lieu de travail a par ailleurs motivé le développement de solutions recommandant à l'utilisateur des collaborateurs experts dans des domaines pour lesquels il manque de connaissances\cite{DBLP:journals/procedia/BehamKLL10}.\\
~\\Une autre approche parfois employée consiste à générer des messages personnalisés encourageant l'utilisateur à se mettre en situation d'apprentissage. Ce type de solution est notamment utilisé en conjonction sur des plate-formes mobiles. \cite{DBLP:conf/wmte/PetersenM06}. \\
~\\Dans le contexte d'une plate-forme de e-learning comportant des fonctionnalités d'interaction entre différents utilisateurs, il devient envisageable de suggérer à l'utilisateur d'effectuer de telles activités, comme poster un sujet sur un forum intégré à la plate-forme après la consultation d'une ressource \cite{Liou:2014:CPL:2617848.2617854}.

\subsection{Evaluation des différentes solutions}
Il est difficile de comparer les différents systèmes des recommandations proposés par tous ces articles. Il y a plusieurs raisons à cela. Tout d'abord, la plupart d'entre eux ne sont qu'à l'état de prototype et ont peu été testés. D'autre part, les systèmes étudiés portent sur des domaines très variés, et comme on l'a vu plus haut, ils ne recommandent pas tous le même type de ressource. Certains systèmes recommandent des ressources telles que de la documentation ou des exercices, alors que d'autres recommandent des personnes, etc. Enfin, les méthodes et jeux de tests varient grandement d'une étude à l'autre.\\

L'étude de Verbert et al. \cite{DBLP:journals/tlt/VerbertMOWDBD12} soulève ces points et rapporte également quelques résultats de tests. Les évaluations ont pu porter sur l'efficacité des systèmes à améliorer l'apprentissage, sur la pertinence du contenu recommandé, ou encore sur l'utilité et des différents systèmes. La plupart du temps, les systèmes semblent effectivement plus efficaces que des méthodes plus traditionnelles, mais que cela peut dépendre du domaine. La pertinence du contenu a pu être évaluée avec là encore diverses méthodes, la plupart des système évalués renvoyant de meilleurs résultats que des systèmes de filtrage collaboratif classiques. Le sujet de l'utilité du système a semble-t-il été mesuré pour plus de systèmes que les autres points, et là encore avec des méthodes différentes. Si les différentes études rapportent que les différents systèmes ont tous été perçus comme utiles, beaucoup soulèvent également plusieurs problèmes d'ergonomie, et des difficultés d'adaptation de la part des utilisateurs.\\




\section{Conclusion}

% references section

% can use a bibliography generated by BibTeX as a .bbl file
% BibTeX documentation can be easily obtained at:
% http://www.ctan.org/tex-archive/biblio/bibtex/contrib/doc/
% The IEEEtran BibTeX style support page is at:
% http://www.michaelshell.org/tex/ieeetran/bibtex/
\bibliographystyle{IEEEtran}
\bibliography{IEEEabrv,./ref.bib}

\end{document}
