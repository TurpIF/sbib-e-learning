\documentclass[conference]{./sty/IEEEtran}
% \documentclass[a4]{article}

% pour les accents utilisés en français
\usepackage[utf8]{inputenc}
\usepackage[T1]{fontenc}

% pour inclure joliment un algorithme
\usepackage{url}
\usepackage[a4paper,pagebackref,bookmarksnumbered]{hyperref} % ps2pdf
% pour inclure joliment un algorithme
\usepackage{algorithmic}

\usepackage{array}
\usepackage{booktabs}

% Si Latex ne fait pas correctement la césure de certains mots,
% indiquez les emplacements corrects pour les tirets dans ces mots
\hyphenation{op-tical net-works semi-conduc-tor}


\begin{document}

\title{Synthèse bibliographique}


\author{
\IEEEauthorblockN{Jérémy Edert, Mathis Paul, Pierre Turpin}
\IEEEauthorblockA{INSA de Lyon}
}

\maketitle

\tableofcontents

\section{Introduction}

Le e-learning se développe de façon importante dans un grand nombre de
domaines, notamment en entreprise pour la formation continue des personnels.
Dans le cadre du projet de synthèse bibliographique, nous cherchons à étudier
et comparer les différentes solutions existantes dans le domaine des systèmes
de recommendation appliqués au e-learning. \\

L'ensemble des fiches de lecture ci-jointent en annexe nous a permis d'établir
cette synthèse et résument les papiers de recherche lues. Ces articles ont été
repris de différentes conférence et journaux et se concentre sur les trois
thèmes principaux que nous aborderont : la recherche de contenu, l'utilisation
du contexte et des informations de l'utilisateur lors du ciblage, puis
l'application dans le domaine du e-learning. \\

La plupart des articles lus et résumés sont tirés des références de
\cite{DBLP:journals/tlt/VerbertMOWDBD12}. En effet ce dernier regroupe déjà
beaucoup de résultats et de méthodes de recherche contextuelle de documents
pédagogique pour du e-learning. \\

% TODO bullshiter un peu plus

\section{Recherche de contenu}

Le problème de l'application souhaité repose sur la recherche de données et la
proposition de solutions à l'utilisateur. \\

\section{Utilisation du contexte et des informations de l'utilisateur lors du ciblage}

Pour personnaliser et fiabiliser les résultats de recherche pour un
utilisateur, le contexte et les données utilisateurs sont acquis et modifie le
rang des éléments. \\

\subsection{Modélisation du contexte}

Le besoin de modéliser proprement et strictement ce genre d'information est
donc devenu nécessaire. Beaucoup de chercheurs étudient et publient des papiers
sur la notion de contexte et sur sa définition la plus exacte possible
\cite{DBLP:journals/tlt/VerbertMOWDBD12},
\cite{DBLP:reference/rsh/AdomaviciusT11},
\cite{DBLP:journals/aim/AdomaviciusMRT11}. \\

La notion de contexte a été étudier dans différentes sciences (psychologie,
intelligence artificielle, modélisation cognitive, ...) et chacune de ces
sciences expriment un point de vue différents
\cite{DBLP:reference/rsh/AdomaviciusT11}. C'est pourquoi nous ne pouvons pas
donner une liste exhaustive des éléments pouvant définir cette notion. Nous
nous baserons principalement sur un nombre fini d'aspects et d'attributs les
plus utilisés dans le cadre des applications de recommendation avec contexte
pour le e-learning. \\


\subsubsection{Connaissance et acquisition du contexte}
~\\
Chaque attributs formant le contexte doit avoir sa méthode d'acquisition, de
mise à jour (s'il y en a une) et sa "valeur par défaut" lors de données
manquantes. En effet, toutes ces caractéristiques peuvent être statique ou
dynamique et observable ou peu voir non observable
\cite{DBLP:journals/aim/AdomaviciusMRT11}. Le tableau
\ref{tab:update_observability_context} reprend les caractéristique possible
pour un attribut de contexte. \\

\begin{table}
  \caption{\label{tab:update_observability_context} Information sur le contexte}
  \begin{tabular}{|p{0.20\linewidth}|p{0.20\linewidth}|p{0.20\linewidth}|p{0.20\linewidth}|}
    \hline
    ~ & Totalement observable & Partiellement observable & Non observable \\ \hline
    Statique & Connaissance totale & Connaissance partielle et statique & Connaissance latent \\ \hline
    Dynamique & Connaissance totale dynamique & Connaissance partielle et dynamique & Aucune connaissance \\ \hline
  \end{tabular}
\end{table}

Les attributs statiques sont des simplifications que le système peut se
permettre car la donnée est considéré comme figé dans le temps le long de
l'utilisation de l'application par un utilisateur. Par exemple une date de
naissance, une identité sont des données fixe pour un utilisateur. Lorsque ces
attributs sont pleinement observables, le système peut considérer les connaître
entièrement. C'est les meilleurs cas possibles. S'ils sont partiellement
observable ou peu ou non observable, alors cela implique un retard dans
l'obtention des données plus ou moins grand en fonction de l'observabilité des
attributs. Cependant une fois totalement observés, le système connaît ces
attributs. \\

\subsubsection{Utilisateur}
~\\
Pour fiabiliser les résultats de recherche pour un utilisateur, le système doit
prendre en compte les informations de l'utilisateur. \\

\section{Application spécifique dans le domaine du e-learning}

\section{Conclusion}

% references section

% can use a bibliography generated by BibTeX as a .bbl file
% BibTeX documentation can be easily obtained at:
% http://www.ctan.org/tex-archive/biblio/bibtex/contrib/doc/
% The IEEEtran BibTeX style support page is at:
% http://www.michaelshell.org/tex/ieeetran/bibtex/
\bibliographystyle{IEEEtran}
\bibliography{IEEEabrv,./ref.bib}

\end{document}
