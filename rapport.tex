\documentclass[conference]{./sty/IEEEtran}
% \documentclass[a4]{article}

% pour les accents utilisés en français
\usepackage[utf8]{inputenc}
\usepackage[T1]{fontenc}

% pour inclure joliment un algorithme
\usepackage{url}
\usepackage[a4paper,pagebackref,bookmarksnumbered]{hyperref} % ps2pdf
% pour inclure joliment un algorithme
\usepackage{algorithmic}

% Si Latex ne fait pas correctement la césure de certains mots,
% indiquez les emplacements corrects pour les tirets dans ces mots
\hyphenation{op-tical net-works semi-conduc-tor}


\begin{document}

\title{Synthèse bibliographique}


\author{
\IEEEauthorblockN{Jérémy Edert, Mathis Paul, Pierre Turpin}
\IEEEauthorblockA{INSA de Lyon}
}

\maketitle

\tableofcontents

\section{Introduction}

Le e-learning se développe de façon importante dans un grand nombre de
domaines, notamment en entreprise pour la formation continue des personnels.
Dans le cadre du projet de synthèse bibliographique, nous cherchons à étudier
et comparer les différentes solutions existantes dans le domaine des systèmes
de recommendation appliqués au e-learning. \\

L'ensemble des fiches de lecture ci-jointent en annexe nous a permis d'établir
cette synthèse et résument les papiers de recherche lues. Ces articles ont été
repris de différentes conférence et journaux et se concentre sur les trois
thèmes principaux que nous aborderont : la recherche de contenu, l'utilisation
du contexte et des informations de l'utilisateur lors du ciblage, puis
l'application dans le domaine du e-learning. \\

La plupart des articles lus et résumés sont tirés des références de
\cite{DBLP:journals/tlt/VerbertMOWDBD12}. En effet ce dernier regroupe déjà
beaucoup de résultats et de méthodes de recherche contextuelle de documents
pédagogique pour du e-learning. \\

% TODO bullshiter un peu plus

\section{Recherche de contenu}

\section{Utilisation du contexte et des informations de l'utilisateur lors du ciblage}

\section{Application spécifique dans le domaine du e-learning}

\section{Conclusion}

% references section

% can use a bibliography generated by BibTeX as a .bbl file
% BibTeX documentation can be easily obtained at:
% http://www.ctan.org/tex-archive/biblio/bibtex/contrib/doc/
% The IEEEtran BibTeX style support page is at:
% http://www.michaelshell.org/tex/ieeetran/bibtex/
\bibliographystyle{IEEEtran}
\bibliography{IEEEabrv,./ref.bib}

\end{document}
